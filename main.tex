\documentclass{article}
\usepackage[UTF8]{ctex}
\usepackage{geometry}
\usepackage{natbib}
\usepackage{float}
\geometry{left=3.18cm,right=3.18cm,top=2.54cm,bottom=2.54cm}
\usepackage{graphicx}
\pagestyle{plain}	
\usepackage{setspace}
\usepackage{caption2}
\usepackage{datetime} %日期
\renewcommand{\today}{\number\year 年 \number\month 月 \number\day 日}
\renewcommand{\captionlabelfont}{\small}
\renewcommand{\captionfont}{\small}
\begin{document}
	
	\begin{figure}
		\centering
		\includegraphics[width=8cm]{upc}
		
		\label{figupc}
	\end{figure}
	
	\begin{center}
		\quad \\
		\quad \\
		\heiti \fontsize{45}{17} \quad \quad \quad 
		\vskip 1.5cm
		\heiti \zihao{2} 《计算科学导论》课程总结报告
	\end{center}
	\vskip 2.0cm
	
	\begin{quotation}
		% 	\begin{center}
		\doublespacing
		
		\zihao{4}\par\setlength\parindent{7em}
		\quad 
		
		学生姓名:\underline{\qquad  雷鹏辉\qquad \qquad}
		
		学\hspace{0.61cm} 号:\underline{\qquad 1907010220\qquad}
		
		专业班级:\underline{\qquad 人工智能1901 \qquad  }
		
		学\hspace{0.61cm} 院:\underline{计算机科学与技术学院}
		% 	\end{center}
		\vskip 2cm
		\centering
		\begin{table}[h]
			\centering 
			\zihao{4}
			\begin{tabular}{|c|c|c|c|c|c|c|}
				% 这里的rl 与表格对应可以看到,姓名是r,右对齐的;学号是l,左对齐的;若想居中,使用c关键字。
				\hline
				课程认识 & 问题思 考 & 格式规范  & IT工具  & Latex附加  & 总分 & 评阅教师 \\
				30\% & 30\% & 20\% & 20\% & 10\% &  &  \\
				\hline
				& & & & & &\\
				& & & & & &\\
				\hline
			\end{tabular}
		\end{table}
		\vskip 2cm
		\today
	\end{quotation}
	
	\thispagestyle{empty}
	\newpage
	\setcounter{page}{1}
% 在这之前是封面,在这之后是正文
\section{引言}
\large{}从开学至今,已经上过很多节的计算机导论课程了,我主要对漏洞扫描进行了探知,而对此也有一个初步的认识。在此我对我的认知进行相应阐述并深一步的探讨,以概括我的认知并表达我的看法。并且对这次的课堂演讲进行一次总结,以及关于课堂上的问题进行深一步的回答。这次的报告主要关于以下几个方面(1)对计算机导论的课程总结(2)对漏洞扫描的进一步探知(3)对课堂问题进行回答与探讨

\section{对计算科学导论这门课程的认识、体会}
计算机科学导论为本专业的一门专业课程,通过学习使我对计算机系统有了初步了解,对计算机专业课程有了全面了解。关于计算机科学导论主要有以下课程:C语言程序设计、计算机组成原理、编译原理、离散数学、数字逻辑、数值分析、数据结构、操作系统、微机原理及汇编语言、计算机网络、计算机系统结构、软件工程、面向对象程序设计电路原理、计算机英语等。而所谓的计算机科学,一般指的是数据结构,组成原理,操作系统,编译原理等计算机内部实现机制。而我们这个专业的主要学习计算机科学与技术方面的基本理论和基本知识,接受应用计算机的基本训练,具有开发计算机系统的基本能力。对于此我查阅的相关文献也有所提及\citep{jisuanjidaolun}。
\subsection{计算机的发展史}
计算机的发展不是一蹴而就的,而是经过漫长的历史过程。1946年由冯诺依曼发明的ENIAC是世界上第一台电子计算机,它的产生明确了计算机的五大部分:运算器、控制器、存储器、输入设备、输出设备,并使用二进制运算代替了原来十进制运算,对今后计算机的发展有着巨大的影响。随后又经历了第一代计算机(电子管1951—1959)、第二代计算机(晶体管1959—1963)、第三代计算机(集成电路1964—1975)、第四代计算机(超大规模集成电路式微处理器1975—至今)的四次改革,使得计算机走进寻常人家,适应了社会的需要。
\subsection{计算机学科的发展主线}
从科学哲学的角度,将计算科学的学科内容按照基础理论、基本开发技术、应用以及它们与硬件设备联系的紧密程度可分成三个层面。

第一层面是计算科学的应用层,它包括人工智能应用与系统、信息、管理与决策系统,移动计算、计算可视化、科学计算等计算机应用的各方向。其中,人工智能应用与系统涵盖人工智能,机器人,神经元计算,知识工程,自然语言处理与机器翻译,自动推理等方向;信息、管理与决疲系统涵盖数据库设计与数据管理技术,数据表示与存储(包括多媒体技术),数据与信息检索,管理信息系统,计算机辅助系统,决策系统等方向;计算可视化涵盖计算机图形学,计算几何,模式识别与图像处理等方向。

第二层面是计算科学的专业基础层,它是为应用层提供技术和环境的一个层面,包括软件开发方法学,计算机网络与通信技术,程序设计科学,计算机体系结构,电子计算机系统基础。其中,软件开发方法学涵盖顺序、并行与分布式软件开发方法学,如软件工程技术,软件开发工具和环境等方向;计算机网络与通信技术涵盖计算机网络互联技术,数据通信技术,以及信息保密与安全技术等方向;程序设计科学涵盖数据结构技术,数值与符号计算,算法设计与分析(包括并行与分布式算法设计与分析),程序设计语言,程序设计语言的文法与语义(指程序设计语言的文法与语义描述),程序设计方法学,程序理论等方向;电子计算机系统基础涵盖数字逻辑技术,计算机组成原理,故障诊断与器件测试技术,操作系统.编译技术,数据库系统实现技未,容错技术等方向。

第三层面是计算科学的基础层,它包括计算的数学理论,高等逻辑等内容。其中,计算的数学理论涵盖可计算性(递归论)与计算复杂性理论,形式语言与自动机理论,形式语义学(主要指代数语义,公理语义等),Petri网理论等方向;高等逻辑涵盖模型论,各种非经典逻辑与公理集合论等方向。

支撑这三个层面的是计算科学这一学科的理工科基础科目,包括物理学(主要是电子技术科学)、基础数学(含离散数学)等。
\subsection{如何学好计算机科学这个领域}
首先,英语是必须学好的一门学科。由于计算机是西方第一个制作出来的,所以它的编译语言是英语,包括一些重要的专业知识,也需要极高的语言要求。所以一旦英语没有学好,这门学科就不可能掌握。其次就是离散数学。这是计算机专业的一门非常重要的基础专业知识。所以在学习时必须一丝不苟,上课要认真听课,课后也要及时复习。只有学好了离散数学,才能学好计算机。接下来就是计算机语言的学习了。这些虽然以前从没有接触过,但是经过一个学期的学习后,对这些知识有了一定的了解,所以以后学习起来即使会有一些困难也会努力克服。总而言之,要付出百分之百的努力学习专业知识,打好基础,同时还要提高自己各方面能力。
\subsection{计算机产业的发展前景}
计算机产业作为工业革命的产物,在20世纪的出现已经极大地改变了整个世界的面貌,深刻影响并仍将继续影响世界各国政治、经济、军事、文化、环境格局,人类的生存前景和生活质量。而在我国主要是软件的发展,下面我们重点讨论软件产业在中国的发展前景。众所周知,软件的开发首先是一项高智力的活动,软件产业的发展既有生产成本低,产品高附加值,高收益的特点,也有产品寿命短,升级代换快,市场变化快,投资风险大的特点。总结过去我们在发展软件产业方面的经验和教训,对今后更好的发展软件产业是十分有益的。我们过去的主要问题是没有按照软件产业发展的规律行事,过多的依赖科研机构。现在,越来越多的人已经认识到了我们处于被动的这一现状,并开始着手改革。首先,在一些高校中对人才的培养加大了基础课程和专业基础课程的改革力度,转变人才的培养观念,改革旧的教学模式。其次,产业投资主体发生了明显的变化,国家开始逐渐转到改善投资环境,扶持重点企业。这是正确的决策,相信随着国家软件政策的调整,随着对高校投入的增加,实验室的改善,随着重点企业软件自主研发与开发能力的增强,我国软件产业一定会在不远的将来赶上世界先进水平。
\section{进一步的思考(漏洞扫描)}
关于这次的课题演讲主要是探讨了漏洞扫描并对此进行了深一步探知。然后分为主机扫描和网络扫描讲解。
\subsection{漏洞扫描的概念}
漏洞又称为脆弱性,它是计算机系统在软件、硬件及协议
的具体实现或系统策略存在的缺陷。一旦发现网络和主机之间
存在缺陷,漏洞就将会乘虚而入,给电脑及其上面所存储的信
息带来危害。但是,不同的漏洞需要不同的应答数据和特征码,
只有互相匹配的应答数据特征码才能够形成网络漏洞。因此,
对特征码进行扫描就成了保障网络安全的重要手段。

安全是相对的,任何系统都存在漏洞,从计算机网络产生
的那天开始,漏洞也就存在,但是大多数的攻击者利用的都是
常见的漏洞。因此,对已知的漏洞进行扫描远比对无知的漏洞
进行扫描要轻松、简单的多,而且花费的人力、物力、财力也
比较的少。目前为止,在人们所发现的所有漏洞中,一般将其
划分为以下两种:一种是应用软件漏洞,它主要存在于为系统
提供网络服务软件的服务器上,如WWw服务漏洞、SMTP服
务漏洞、FTP服务漏洞、Telnet服务漏洞等。而另一类是操作
系统漏洞,它是Windows系统中的一些常见的RPC漏洞。因
此,我们必须在这次漏洞被不良分子发现之前,对其进行及时
的修补。尽管你不是专业人士,但是也不用担心,因为目前市
面上所出售的漏洞扫描是能够将以上两种漏洞扫描出来的。

但是,对于漏洞扫描软件,它也存在着自己的优劣,它既
可以为我们包围网络安全提供便利,也可以为黑客进入网络提
供便利。不同的人不同的用途,黑客利用扫描工具入侵,管理
员使用它来防范。因此,要正确的对待扫描工具,并让其发挥
最大的作用。\citep{loudongsaomiao}
\subsection{主机扫描}
主机扫描核心模块驻留于目标主机,它既可以接受远程
用户通过管理控制平台传来的启动命令及配置参数来完成一
次扫描过程,也可以根据本地管理员的要求完成扫描功能,其
工作原理如图\ref{fig:23}所示。
    \begin{figure}[H]
	\centering
	\includegraphics[scale=1.0]{zhujisaomiao}
	\caption{主机扫描工作原理}
	\label{fig:23}
    \end{figure}
用户通过图形用户界面的管理平台管理整个扫描过程。
在管理平台上,用户首先进行初始化配置,包括对扫描插件的
选择和扫描时所需参数的配置等。然后启动扫描,执行各个
扫描插件。在扫描插件的执行过程中,管理员可以对扫描过
程进行干预(暂停或取消正在进行的扫描)。当完成扫描后,
形成正式的报告,此报告可以根据不同情况或者通过通信接
口反馈给远程管理平台,或者直接反馈给本地管理平台,也可
以对其进行进一步的分析,形成更有意义的预警信息,用于指
导系统管理员进行决策。\citep{Unix}

特点:

(1)操作方便:系统不但提供了友好的图形用户界面,而
H还提供了两种管理模式,所以无论你在本地还是远程计算
机都可以方便的通过不同的插件选择来控制扫描行为。

(2)功能强大:上机漏洞扫揣的目的就是发现系统脆弱点,
并指导系统管理员重新配置系统,使系统更加安全。通过扫
描,我们不仅给出了系统的漏洞信息,而且就许多脆弱点提出
了修复建议及增强措施,如账号的安全策略、超级用户的控制
和监测、su命令的使用、suid/guid文件的设置、口令安全的保
障、文件/用录最小权限原则、关闭不必要的服务、网络服务的加强、主机间的共享以及对系统定时升级等。

(3)扫描内容全面,速度快:本系统致力于分析出所有已
知的系统脆弱点,并通过设计扫描插件实现对主机的探测分
析,而要达到这目的就必须全面地收集各方而的信息。信
息的收集主要通过以下两个渠道:第1,来自各大厂商的信息;
第2,来自第三方的信息,这些信息一般是由关心安全问题的
机构和个人免费提供的,通过多种渠道反复、及时地收集漏
洞信息并设计相应插件,使得扫描更全面,而各插件的编写采
用C语言或shell脚本语言,执行速度更快。

\subsection{网络扫描}
基于网络的扫描是从外部攻击者的角度对网络及系统架构进行的扫描,主要用于查找网络服务和协议中的漏洞,如可以查找网络中运行的SNMP服务的漏洞。基于网络的扫描可以及时获取网络漏洞信息,有效的发现那些网络服务和协议的漏洞如DNS服务和底层协议的漏洞;同时能够有效的发现那些基于主机的扫描不能发现的网络设备漏洞,如路由器、交换机、远程访问服务和防火墙等存在的漏洞。\citep{Xu2010CANVuS}



整个网络扫描器的工作原理是:当用户通过控制平台发出了扫描命令之后,控制平台即向扫描模块发出相应的扫描请求,扫描模块在接到请求之后立即启动相应的子功能模块,对被扫描主机进行扫描。通过对从被扫描主机返回的信息进行分析判断,扫描模块将扫描结果返回给控制平台,再由控制平台最终呈现给用户。如图\ref{fig:24}。
    \begin{figure}[H]
	\centering
	\includegraphics[scale=0.8]{wanluo}
	\caption{网络扫描流程图}
	\label{fig:24}
    \end{figure}
特点:

(1)服务器模式下具备良好的用户管理、扫描目标地址记录、源地址记录、扫描时间记录,用户无法删除服务器上的任何记录,完全可以避免滥用扫描器这种双面刃工具。
(2)由于知识库是基于数据库的,可以实现在线更新,新功能的增加和模块更新方便而且快速。
(3)数据库更新速度快。
(4)扫描速度快,由于服务器的资源优势,以及系统任务分配功能,能对网络上的服务器进行快速的安全检测,节省时间。
(5)提供很多外国安全检测产品所不能发现的技术漏洞检测模块。一些中国黑客发现的对中文系统危害性大的安全漏洞未被公布到Internet上,但已经集成到扫措服务系统中,适合中国国情。
\section{课堂问题回答}
近几年漏洞扫描安全发现事件相对与几年前多了还是少了?为什么?

少了,对于这个问题可能有以下几点原因:

1.近几年来windows安全系统提高了。举个例子:前几年,一种信息泄露类型的恶意软件(或恶意代码)不断增加,用于泄露个人数据、信用信息、财务信息等,而且这种形式也在迅速变化。而windows对此进行了针对方案,在文献New security rules to raise Windows  \citep{Fisher2003New}中也有体现,诸如这句话“Improvement of security of its products; Impacts of the security system on the upcoming products.”

2.随着科技的发展以及人们对网络病毒的重视,越来越多的杀毒软件和漏洞扫描系统被研制出来,而大多的软件或是系统都是免费的,因此,许多病毒事件也大有减少。
\section{总结}
漏洞扫描将越来越被重视,因为未来网络发展的基础便是网络的安全,一个计算机企业的成立必然有漏洞扫描系统,它是战争的基石也是战争胜利的保障。未来必会更加智能化。一学期的计算机导论课程就这么结束了,未来的计算机领域也会更加繁荣。新时代是IT行业的蓬勃发展时代,作为即将走出社会的我们应该抓住时代发展的主旋律,在此行业中探寻自身发展的道路。计算机科学如同参天大树,里面的分支犹如它的根茎庞大而复杂。经过这门课程的学习,使得我初步认识了计算机这一学科,它是深奥的,但同时也是有着巨大作用的一门学科。希望在以后的学习和工作当中能接触到更深层次的知识。
\par

\newpage
\section{附录}
\begin{itemize}
    \item Github个人网址:https://github.com/leipenghui/leipenghui.github.io/upload/master
     \begin{figure}[H]
    \centering
    \includegraphics[scale=0.2]{gitub}
    \caption{gitub网页截图}
    \label{fig:1}
      \end{figure}
    \item 注册观察者:https://user.guancha.cn/main/index?s=fwdhsy
    \begin{figure}[H]
    	\centering
    	\includegraphics[scale=0.2]{guanchazhe}
    	\caption{观察者网页截图}
    	\label{fig:3}
    \end{figure}
    学习强国:https://www.xuexi.cn/
    \begin{figure}[H]
    	\centering
    	\includegraphics[scale=0.2]{xuexiqiangguo}
    	\caption{学习强国网页截图}
    	\label{fig:4}
    \end{figure}
    哔哩哔哩:https://space.bilibili.com/484178878
    \begin{figure}[H]
    	\centering
    	\includegraphics[scale=0.2]{bilibili}
    	\caption{哔哩哔哩网站截图}
    	\label{fig:5}
    \end{figure}
    \item 注册CSDN:https://www.csdn.net/
    \begin{figure}[H]
    	\centering
    	\includegraphics[scale=0.2]{csdn}
    	\caption{CSDN网页截图}
    	\label{fig:6}
    \end{figure}
    博客园账户:https://home.cnblogs.com/
    \begin{figure}[H]
    	\centering
    	\includegraphics[scale=0.2]{bokeyuan}
    	\caption{博客园网页截图}
    	\label{fig:7}
    \end{figure}
    \item 注册小木虫账户:http://muchong.com/bbs/
     \begin{figure}[H]
    	\centering
    	\includegraphics[scale=0.2]{xiaomuchong}
    	\caption{小木虫网页截图}
    	\label{fig:8}
    \end{figure}
\end{itemize}


\hspace*{\fill} \\


\bibliographystyle{plain}
\bibliography{references}


\end{document}
